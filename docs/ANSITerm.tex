\documentclass[10pt, openany]{book}
%
%  Packages to use
%
\usepackage{fancyhdr}
\usepackage{fancyvrb}
\usepackage{fancybox}
%
\usepackage{lastpage}
\usepackage{imakeidx}
%
\usepackage{amsmath}
\usepackage{amsfonts}
%
\usepackage{geometry}
\geometry{letterpaper}
%
\usepackage{url}
\usepackage{gensymb}
\usepackage{multicol}
\usepackage{xcolor}
%
\usepackage{tikz}
\usepackage[pdf]{pstricks}
\usepackage{graphicx}
\DeclareGraphicsExtensions{.pdf}
\DeclareGraphicsRule{.pdf}{pdf}{.pdf}{}
%
% Rules to allow import of graphics files in EPS format
%
\usepackage{graphicx}
\DeclareGraphicsExtensions{.eps}
\DeclareGraphicsRule{.eps}{eps}{.eps}{}
%
%  Include the listings package
%
\usepackage{listings}
%
%  Setup indexes
%
\makeindex[name=type,title=List of Datatypes,columns=3]
\newcommand{\indextype}[1]{\index[type]{#1}}
\makeindex[name=func,title=List of Functions/Procedures,columns=3]
\newcommand{\indexfunc}[1]{\index[func]{#1}}
%
% Macro definitions
%
\newcommand{\operation}[1]{\textbf{\texttt{#1}}}
\newcommand{\package}[1]{\texttt{#1}}
\newcommand{\function}[1]{\texttt{#1}}
\newcommand{\constant}[1]{\emph{\texttt{#1}}}
\newcommand{\keyword}[1]{\texttt{#1}}
\newcommand{\datatype}[1]{\texttt{#1}}
\newcommand{\filename}[1]{\texttt{#1}}
\newcommand{\cli}[1]{\texttt{#1}}
\newcommand{\uvec}[1]{\textnormal{\bfseries{#1}}}
\newcommand{\comment}[1]{{\color{red}{#1}}}
%
\newcommand{\docname}{Users's Manual for \\ ANSI Terminal Interface}
%
% Front Matter
%
\title{\docname}
\author{Brent Seidel \\ Phoenix, AZ}
\date{ \today }
%========================================================
%%% BEGIN DOCUMENT
\begin{document}
%
%  Header's and Footers
%
\fancypagestyle{plain}{
  \fancyhead[L]{}%
  \fancyhead[R]{}%
  \fancyfoot[C]{Page \thepage\ of \pageref{LastPage}}%
  \fancyfoot[L]{Ada Programming}
  \renewcommand{\headrulewidth}{0pt}%
  \renewcommand{\footrulewidth}{0.4pt}%
}
\fancypagestyle{myfancy}{
  \fancyhead[L]{\docname}%
  \fancyhead[R]{\leftmark}
  \fancyfoot[C]{Page \thepage\ of \pageref{LastPage}}%
  \fancyfoot[L]{Ada Programming}
  \renewcommand{\headrulewidth}{0.4pt}%
  \renewcommand{\footrulewidth}{0.4pt}%
}
\pagestyle{myfancy}
%
% Produce the front matter
%
\frontmatter
\maketitle
\begin{center}
This document is \copyright 2025, Brent Seidel.  All rights reserved.

\paragraph{}Note that this is a draft version and not the final version for publication.
\end{center}
\tableofcontents

\mainmatter
%========================================================
\chapter{Introduction}

\section{About the Project}
The intent of this project is to provide assistance in generating ANSI escape sequences for enhancing terminal interfaces.  Most of these are done using string constants, but a few functions and procedures are also defined.

\section{License}
This project is licensed using the GNU General Public License V3.0.  Should you wish other licensing terms, contact the author.

THE SOFTWARE IS PROVIDED "AS IS", WITHOUT WARRANTY OF ANY KIND, EXPRESS OR IMPLIED, INCLUDING BUT NOT LIMITED TO THE WARRANTIES OF MERCHANTABILITY, FITNESS FOR A PARTICULAR PURPOSE AND NONINFRINGEMENT. IN NO EVENT SHALL THE AUTHORS BE LIABLE FOR ANY CLAIM, DAMAGES OR OTHER LIABILITY, WHETHER IN AN ACTION OF CONTRACT, TORT OR OTHERWISE, ARISING FROM, OUT OF OR IN CONNECTION WITH THE SOFTWARE OR THE USE OR OTHER DEALINGS IN THE SOFTWARE.

%========================================================
\chapter{How to Obtain}
This package is currently available on GitHub at \url{https://github.com/BrentSeidel/ANSITerm}

\section{Dependencies}
The only dependencies for this project are the following standard Ada Libraries.
\subsection{Ada Libraries}
The following Ada libraries are used:
\begin{itemize}
  \item  \package{Ada.Calendar}
  \item \package{Ada.Strings.Unbounded}
  \item \package{Ada.Text\_IO}
\end{itemize}

%========================================================
\chapter{Usage Instructions}

\section{Using Alire}
\comment{This package has not yet been submitted to alire.  Until then, use the \keyword{gprbuild} instructions.}

Alire automatically handles dependencies.  To use this in your project, just issue the command ``\keyword{alr with ansiterm}'' in your project directory.  To build the standalone CLI program, first obtain the cli using ``\keyword{alr get ansiterm}''.  Change to the appropriate directory and use ``\keyword{alr build}'' and ``\keyword{alr run}''. 

\section{Using \keyword{gprbuild}}
This is a library of routines intended to be used by some program.  To use these in your program, edit your *\keyword{.gpr} file to include a line to \keyword{with} the path to \keyword{ansiterm\_noalr.gpr}.  Then in your Ada code \keyword{with} in the package(s) you need and use the routines.

%========================================================
\chapter{API Description}
\comment{If the project does not have a public API, this chapter can be omitted.  Otherwise include an API description here.  This would include packages, data types, routines to call, how to instantiate generics, and anything else that would be valuable to someone using the project.}

%========================================================
\chapter{User Interface}
\comment{If there is no user interface, this chapter can be omitted.  Otherwise, if the project has a user interface, put instructions in this chapter.}

%========================================================
\chapter{Other Stuff}
\comment{If there is anything else that should be added, additional chapters may be added as needed.}

%========================================================
\clearpage
%
%  Add indices
%
\addcontentsline{toc}{chapter}{Indices}
\printindex[type]
\printindex[func]
%
%  Add bibliography
%
\nocite{Ada95}
\nocite{Ada2012}
\nocite{Ada2022}
\nocite{xTerm}
\addcontentsline{toc}{chapter}{Bibliography}
\bibliographystyle{plain}
\bibliography{ANSITerm.bib}

\end{document}
